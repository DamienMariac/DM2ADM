  # Fonction pour calculer l'indice de Rand
  rand_index <- function(partition1, partition2) {
    n <- length(partition1)

    # Initialisation des compteurs
    C1 <- 0 # Même cluster dans P et P'
    C2 <- 0 # Différents clusters dans P et P'
    D1 <- 0 # Même cluster dans P mais pas dans P'
    D2 <- 0 # Même cluster dans P' mais pas dans P

    # Boucle sur toutes les paires possibles
    for (i in 1:(n - 1)) {
      for (j in (i + 1):n) {
        same_in_P <- (partition1[i] == partition1[j])
        same_in_P_prime <- (partition2[i] == partition2[j])

        if (same_in_P && same_in_P_prime) {
          C1 <- C1 + 1
        } else if (!same_in_P && !same_in_P_prime) {
          C2 <- C2 + 1
        } else if (same_in_P && !same_in_P_prime) {
          D1 <- D1 + 1
        } else if (!same_in_P && same_in_P_prime) {
          D2 <- D2 + 1
        }
      }
    }

    # Calcul de l'indice de Rand
    rand_index_value <- (C1 + C2) / (C1 + C2 + D1 + D2)

    return(rand_index_value)
  }
